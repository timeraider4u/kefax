% !TeX encoding = UTF-8
% !TeX spellcheck = en_US
\section{Introduction}
\IEEEPARstart{T}his project has been developed as part of my project in software engineering for the master course computer science.
The tool {\it ECCO}
({\it Extraction and Composition for Clone-and-Own})\cite{fischer2014enhancing}
is developed at the ISSE (Institute for Software Systems Engineering \cite{ISSE_URL})
at Johannes Kepler University in Linz, Austria 
and can be found at \cite{ecourl}.
It maps commonalities and differences of existing variants 
of a portfolio to a feature set.
To provide a real-life case study, the Linux Kernel has been chosen as a 
demonstration example for the {\it ECCO} tool 
(the Linux kernel is a pretty well-known example / case study 
and a project with huge impacts in industry and research).
At the moment, plain C and C++ programs are not supported by {\it ECCO} yet.
Therefore, an importer should be written which is
able to use the output of C parsers to extract the relevant information
for {\it ECCO}. 
The  KeFaX project should provide an importer for the Linux kernel to the ECCO tool. 
As a result,
this project is supposed to parse the Linux .config file, 
set-up the minimal infrastructure of source code for the modules and 
parse the source code. At the end, this project should create an ``input tree'' data
structure. 
Various approaches exist which have been evaluated
for their suitability to the given task. 
This paper will present the steps taken so far.
\\ \ \\
\hfill May 09, 2016
