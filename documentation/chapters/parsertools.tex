% !TeX spellcheck = en_US
% !TeX encoding = UTF-8
\section{Compiler generators}\label{sec:compilercompilers}
Compiler compilers are based on processing formal grammars. 
The research in theoretic computer science has led to
the automation of generating scanners and parsers out of an
existing formal description text of the programming language
(wherefore often the EBNF - extended Backus-Naur-Form - is used
as a notation).
Compiler generators like Yacc/Lex (which have GNU open source
implementations called Bison/Flex) have been around for quite some time.
COCO/R\cite{COCOR} is another compiler engine which has been developed at the SSW
institute at the JKU in Linz. 
One disadvantage of compiler compilers is the mixture of multiple languages 
(one language for describing the scanning/parsing process mixed
with the source code statements for the resulting compiler).
Another disadvantage is that often C/C++ EBNF descriptions lack
language features like ''namespaces'' or others.
But the worst matter is that all information about preprocessor statements
in C and C++
are lost because EBNFs can not deal with include or define macros.
Therefore, it is not possible to find out in which scope a function or variable
has been declared originally and where it has been defined.
\\ \ \\ 
EBNF profiles of various programming languages can be found 
in the grammarware Github repository\cite{Grammarzoo}.

